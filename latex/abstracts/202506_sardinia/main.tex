\documentclass[a4paper,10pt]{article}

\usepackage{sig33,graphicx}

\begin{document}
%
% Please underline the presenting author and give
% at least the email address of that author.
%
\title{Flow transition over gaps}
%
\author{\underline{Víctor Ballester Ribó}$^1$, Spencer Sherwin$^1$, Yongyun Hwang$^1$}
\affiliation{$^1$Department of Aeronautics, Imperial College London, UK}
\maketitle

%
% Provide a short summary of the paper
%

Aircraft wings are not perfectly smooth; they often feature spanwise gaps due to structural components or control surface discontinuities. These gaps can significantly influence boundary layer transition, potentially affecting aerodynamic performance and efficiency. In this study, we investigate how these gaps interact with the transition process, with a particular focus on the generation and amplification of Tollmien-Schlichting (TS) waves~\cite{schmid_henningson_2001}. Specifically, we examine the role of gap-induced modes in triggering or enhancing natural boundary layer instabilities. The analysis is conducted using high-fidelity numerical simulations based on the spectral/hp element method, implemented in the open-source framework Nektar++~\cite{nektar}. The findings contribute to a deeper understanding of transition mechanisms in the presence of geometric discontinuities, which is crucial for designing more aerodynamically efficient wings.
\newline



\begin{thebibliography}{1}
\bibitem{schmid_henningson_2001}
P.~J. Schmid, and D.~S. Henningson.
\newblock {\em Stability and Transition in Shear Flows}.
\newblock Springer, Berlin, 2001.

\bibitem{nektar}
Nektar++.
\newblock https://www.nektar.info/.
\newblock Accessed: 24/02/2025.
\end{thebibliography}
\end{document}
