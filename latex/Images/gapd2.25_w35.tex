\documentclass{standalone}
\usepackage{tikz}
\usepackage{pgfplots}
\usetikzlibrary{patterns,decorations.markings,backgrounds}
\usetikzlibrary{arrows.meta}
\usepgfplotslibrary{colorbrewer,colormaps}

\pgfplotsset{compat=1.18}

\def\image{d2.25_w35white.png}
\def\secondimage{d2.25_w35whiteunscaled.png}

\def\xfirstedge{-5.6}
\def\yfirstedge{-1.42}
\def\xsecondedge{-1.57}
\def\ysecondedge{-2.5}
\def\xmin{-6.075}
\def\xmax{6.075}
\def\ymax{4.6}
\def\ymin{-4.88}

\def\scale{1.5}
\pgfplotsset{colormap/RdBu}
\pgfplotsset{colormap/winter}

% (xmin,y1st edge)------(x1st edge, y1st edge)                         ---------------(xmax, y1st edge)
% 	                   |					     |
% 	                   |                                         |	 			     
% 			   |					     |
% 	                   |					     |
% 	                   ----------------------------------(x2nd edge, y2nd edge)

\def\ysmallgap{3.37}
\def\xsmallgap{-3.59}
\def\depth{0.21}
\def\width{3.18}


\begin{document}
\begin{tikzpicture}
	\clip (\xmin,\ymin) rectangle (\xmax,\ymax);

	% import .png figure
	\node[inner sep=0pt] (gap) at (0,0)
	{\includegraphics[width=\textwidth]{\image}};
	\draw[ultra thick] (\xmin,\yfirstedge) -- (\xfirstedge,\yfirstedge) -- (\xfirstedge,\ysecondedge) -- (\xsecondedge,\ysecondedge) -- (\xsecondedge,\yfirstedge) -- (\xmax,\yfirstedge);

	\node[inner sep=0pt,yshift=1.0cm] (gap2) at (gap.north)
	% trim = left bottom right top
	{\includegraphics[trim=0cm 31cm 0cm 20cm, clip, width=0.6\textwidth]{\secondimage}};
	\draw[thin] (\xsmallgap-0.3,\ysmallgap) -- (\xsmallgap,\ysmallgap) -- (\xsmallgap,\ysmallgap-\depth) -- (\xsmallgap+\width,\ysmallgap-\depth) -- (\xsmallgap+\width, \ysmallgap) -- (\xsmallgap+\width+4.05,\ysmallgap);

	\def\lengthWall{0.4}
	\foreach \x in {-5.7,-1.1,-0.6,...,6.5} {
			\draw[thick] (\x,\yfirstedge) -- ++(225:\lengthWall); % Wall hatching
		}
	\foreach \x in {-5.6,-5.2,...,-1.47} {
			\draw[thick] (\x,\ysecondedge) -- ++(225:\lengthWall); % Wall hatching
		}
	\draw[thick] (\xfirstedge,-1.7) -- ++(225:\lengthWall); % Wall hatching
	\draw[thick] (\xfirstedge,-2.1) -- ++(225:\lengthWall); % Wall hatching
	

	\def\secondlengthWall{0.06}
	\foreach \x in {-3.85,-3.77,...,-3.59,-0.3,-0.22,...,3.7} {
			\draw[very thin] (\x,\ysmallgap) -- ++(225:\secondlengthWall); % Wall hatching
		}
		\foreach \x in {-3.57,-3.49,...,-0.4} {
			\draw[very thin] (\x,\ysmallgap-\depth) -- ++(225:\secondlengthWall); % Wall hatching
		}
		\draw[very thin] (\xsmallgap,\ysmallgap-\depth/4) -- ++(225:\secondlengthWall); % Wall hatching
		\draw[very thin] (\xsmallgap,\ysmallgap-\depth*0.65) -- ++(225:\secondlengthWall); % Wall hatching

	\pgfplotscolorbardrawstandalone[
		colormap={example}{
				indices of colormap=(10, 9,8 ,7,6,5,4,3,2,1,0 of RdBu), % inversion of RdBu -> BuRd
			},
		point meta min=-0.08,
		point meta max=0.08,
		colorbar horizontal,
		samples=5,
		scaled x ticks=false,
		colorbar style={
			xtick={-0.08,-0.04,...,0.08},
				xticklabel style={
						/pgf/number format/fixed,
						/pgf/number format/precision=3,
						scale=\scale,
					},
				xlabel={$v$},
				xlabel style={at={(0.5,-1.8em)}, 	scale=\scale,anchor=north},
				at={(gap.south)},
				yshift=-0.4cm,
				anchor=center,
				width=7cm,
			},
	]
\end{tikzpicture}
\end{document}
