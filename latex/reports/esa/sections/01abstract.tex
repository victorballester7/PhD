\begin{abstract}
	This study investigates the transition of flow over surface gaps in two-dimensional incompressible laminar boundary layers, with particular emphasis on the influence of small surface discontinuities on aerodynamic performance. The research is motivated by the increasing demand for fuel-efficient aircraft designs, where even minor geometric irregularities can significantly affect the onset of laminar--turbulent transition and thereby increase aerodynamic drag. Using \texttt{Nektar++}, an open-source high-fidelity spectral/$hp$ element solver, we solve the incompressible Navier–Stokes equations to examine how variations in gap geometry influence boundary-layer stability. Transition prediction is performed using the classical $e^N$-method, which characterizes the amplification of Tollmien--Schlichting waves. Numerical results are validated against experimental data, showing strong agreement. The analysis demonstrates that both gap width and depth critically affect flow stability, with direct implications for surface design in next-generation low-drag aircraft. The study concludes with proposed extensions to compressible flows and three-dimensional configurations to better capture transition phenomena in more realistic flight conditions.
\end{abstract}
