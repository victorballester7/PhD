\documentclass[../main.tex]{subfiles}

\begin{document}
\section{Introduction}

Aircraft wings are not perfectly smooth; instead, they often exhibit small surface discontinuities, commonly referred to as gaps. These gaps may arise due to manufacturing tolerances (e.g., clearances around fasteners such as screws or rivets), structural assembly features (e.g., spanwise joints between skin panels), or the presence of movable control surfaces such as flaps, slats, or ailerons. Although these discontinuities are typically small in scale, they can significantly affect the development of the boundary layer, particularly its transition from laminar to turbulent flow. Since laminar--turbulent transition is a major contributor to skin-friction drag, even localized disturbances can meaningfully influence overall aerodynamic performance.

To date, few studies have systematically investigated the effects of surface gaps on boundary-layer transition. Most of them rely on the $n$-factor method, originally introduced by Smith and Gamberoni~\cite{smithgamberoni} and Van Ingen~\cite{vanIngen1956}. This method uses the relation $n(x) = \log(A(x)/A_0)$ to quantify the exponential growth of small disturbances of initial amplitude $A_0$ as they travel downstream. According to this framework, transition is assumed to occur when the amplification factor $n$ exceeds a critical value, typically between 5 and 12, depending on the free-stream turbulence and other environmental conditions.

ONERA (\emph{Office National d'Études et de Recherches Aérospatiales}) has recently conducted several experimental, numerical, and theoretical studies on the influence of surface gaps on boundary-layer behavior. The work by B\'eguet et al.~\cite{beguetONERA} provides a comprehensive summary of these investigations. Gap geometry is characterized in terms of $w/\delta^*$ and $d/\delta^*$, where $w$ and $d$ denote the width and depth of the gap, respectively, and $\delta^*$ is the displacement thickness of the boundary layer. Their study classifies vortex structures forming within the gap as functions of $w$ and $d$, focusing on gaps with aspect ratios $d/w \in [0.2, 0.7]$. Regarding the $n$-factor, the authors adopt the notation $\Delta N = N - N_0$, where $N$ denotes the $n$-factor for the gapped configuration and $N_0$ corresponds to the smooth flat-plate case. They report two primary effects: first, surface gaps amplify pre-existing Tollmien--Schlichting (TS) waves, as evidenced by increases in $\Delta N$ downstream of the gap; second, the gaps generate localized instabilities, leading to a peak in $\Delta N$ at the gap location. All these results were obtained under incompressible flow conditions.

Zahn and Rist~\cite{zahnRist} performed numerical simulations to examine the influence of deep gaps ($d/w \geq 5$) in compressible boundary-layer flows at a Mach number of $\Ma = 0.6$. They observed non-monotonic behavior of $\Delta N$ as a function of gap depth, with values ranging from 0 to 0.5.

One of the most recent and detailed investigations was conducted by Crouch et al.~\cite{jeffGaps}, who extensively analyzed the amplification of TS waves over gaps with varying widths and depths. They found that for shallow gaps ($d/w < 0.017$), the $\Delta N$ factor depends primarily on the gap depth $d$ and is independent of the width $w$. Conversely, for deep gaps ($d/w > 0.028$), the $\Delta N$ is governed by the gap width. When both $d$ and $w$ are sufficiently large, the TS-wave-induced transition is bypassed entirely, and the flow transitions to turbulence near the downstream edge of the gap. Their results also suggest that the behavior of shallow gaps resembles that of isolated backward-facing steps.

In this work, we investigate the influence of surface gaps on boundary-layer transition using a combination of direct numerical simulations and linear stability theory in the incompressible regime. As a validation step, we aim to reproduce and interpret the experimental findings of Crouch et al.~\cite{jeffGaps}, employing high-order numerical tools capable of resolving the growth of small-scale instabilities. Our ultimate goal is to develop a computational framework that accurately captures transition phenomena induced by surface gaps and can later be extended to compressible and three-dimensional flows (see~\cref{sec:future_work}).

The transition modelling employed in this work is based on the classical $e^N$ approach, in which linear perturbations (typically Tollmien--Schlichting waves) are tracked as they amplify downstream within the laminar boundary layer. Transition is assumed to occur once the integrated disturbance growth reaches a critical $N$ value. To quantify the influence of the gap, we introduce the $\Delta n$ factor, defined as the difference between the $n$-values obtained for the gapped and smooth configurations. We deliberately use the notation $\Delta n$, instead of the conventional $\Delta N$, to distinguish the growth analysis employed in this work from the optimal growth analysis in the $e^N$-method formulation (see~\cref{sec:results}). From a computational standpoint, we solve the incompressible Navier--Stokes equations using the spectral/$hp$ element method, which combines the geometric flexibility of finite elements with the accuracy of spectral methods. All simulations are performed using the open-source \texttt{Nektar++} framework~\cite{nektar}.

This report is organized as follows. In~\cref{sec:theory,sec:lst}, we present the theoretical and numerical foundations of our study, including the numerical implementation, boundary-layer theory, and the $e^N$ transition model. In~\cref{sec:results}, we compare our simulation results with experimental data, analyze the amplification of instability waves over surface gaps, and assess the sensitivity of transition to geometric variations. Finally, in~\cref{sec:future_work}, we outline potential directions for future research, including compressible extensions and full three-dimensional simulations.

\end{document}
